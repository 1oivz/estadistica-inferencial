% Options for packages loaded elsewhere
\PassOptionsToPackage{unicode}{hyperref}
\PassOptionsToPackage{hyphens}{url}
\PassOptionsToPackage{dvipsnames,svgnames*,x11names*}{xcolor}
%
\documentclass[
]{article}
\usepackage{lmodern}
\usepackage{amssymb,amsmath}
\usepackage{ifxetex,ifluatex}
\ifnum 0\ifxetex 1\fi\ifluatex 1\fi=0 % if pdftex
  \usepackage[T1]{fontenc}
  \usepackage[utf8]{inputenc}
  \usepackage{textcomp} % provide euro and other symbols
\else % if luatex or xetex
  \usepackage{unicode-math}
  \defaultfontfeatures{Scale=MatchLowercase}
  \defaultfontfeatures[\rmfamily]{Ligatures=TeX,Scale=1}
\fi
% Use upquote if available, for straight quotes in verbatim environments
\IfFileExists{upquote.sty}{\usepackage{upquote}}{}
\IfFileExists{microtype.sty}{% use microtype if available
  \usepackage[]{microtype}
  \UseMicrotypeSet[protrusion]{basicmath} % disable protrusion for tt fonts
}{}
\makeatletter
\@ifundefined{KOMAClassName}{% if non-KOMA class
  \IfFileExists{parskip.sty}{%
    \usepackage{parskip}
  }{% else
    \setlength{\parindent}{0pt}
    \setlength{\parskip}{6pt plus 2pt minus 1pt}}
}{% if KOMA class
  \KOMAoptions{parskip=half}}
\makeatother
\usepackage{xcolor}
\IfFileExists{xurl.sty}{\usepackage{xurl}}{} % add URL line breaks if available
\IfFileExists{bookmark.sty}{\usepackage{bookmark}}{\usepackage{hyperref}}
\hypersetup{
  pdftitle={Ejercicios Tema 2 - Estimación SOLUCIONES. Taller 3},
  pdfauthor={Ricardo Alberich, Juan Gabriel Gomila y Arnau Mir},
  colorlinks=true,
  linkcolor=red,
  filecolor=Maroon,
  citecolor=blue,
  urlcolor=blue,
  pdfcreator={LaTeX via pandoc}}
\urlstyle{same} % disable monospaced font for URLs
\usepackage[margin=1in]{geometry}
\usepackage{graphicx}
\makeatletter
\def\maxwidth{\ifdim\Gin@nat@width>\linewidth\linewidth\else\Gin@nat@width\fi}
\def\maxheight{\ifdim\Gin@nat@height>\textheight\textheight\else\Gin@nat@height\fi}
\makeatother
% Scale images if necessary, so that they will not overflow the page
% margins by default, and it is still possible to overwrite the defaults
% using explicit options in \includegraphics[width, height, ...]{}
\setkeys{Gin}{width=\maxwidth,height=\maxheight,keepaspectratio}
% Set default figure placement to htbp
\makeatletter
\def\fps@figure{htbp}
\makeatother
\setlength{\emergencystretch}{3em} % prevent overfull lines
\providecommand{\tightlist}{%
  \setlength{\itemsep}{0pt}\setlength{\parskip}{0pt}}
\setcounter{secnumdepth}{5}
\renewcommand{\contentsname}{Contenidos}

\title{Ejercicios Tema 2 - Estimación SOLUCIONES. Taller 3}
\author{Ricardo Alberich, Juan Gabriel Gomila y Arnau Mir}
\date{Curso completo de estadística inferencial con R y Python}

\begin{document}
\maketitle

{
\hypersetup{linkcolor=blue}
\setcounter{tocdepth}{2}
\tableofcontents
}
\hypertarget{estimaciuxf3n-taller-3}{%
\section{Estimación taller 3}\label{estimaciuxf3n-taller-3}}

\hypertarget{ejercicio-1}{%
\subsection{Ejercicio 1}\label{ejercicio-1}}

Supongamos que \(X_1,X_2,\ldots,X_6\) es una muestra aleatoria de una
variable aleatoria normal con media \(\mu\) y varianza \(\sigma^2\).
Hallar la constante \(C\) tal que
\[C\cdot\bigl({(X_1 -X_2)}^2 +{(X_3 -X_4)}^2 + 
 {(X_5 -X_6)}^2\bigr),\] sea un estimador sin sesgo de \(\sigma^2\).

\hypertarget{ejercicio-2}{%
\subsection{Ejercicio 2}\label{ejercicio-2}}

Supongamos que \(\Theta_1\) y \(\Theta_2\) son estimadores sin sesgo de
un parámetro desconocido \(\theta\), con varianzas conocidas
\(\sigma_1^2\) y \(\sigma_2^2\), respectivamente. Demostrar que
\(\Theta =(1-a)\cdot\Theta_1 +a\cdot \Theta_2\) también es insesgado
para cualquier valor de \(a\not=0\).

\hypertarget{ejercicio-3}{%
\subsection{Ejercicio 3}\label{ejercicio-3}}

Sea \(X_1,\ldots,X_{2n}\) una muestra aleatoria simple de una variable
aleatoria \(N(\mu,\sigma)\). Sea: \[
T=C\left({\left(\sum_{i=1}^{2n} X_i\right)}^2- 4 n\sum_{i=1}^{n}
X_{2i} X_{2i-1}\right)
\] un estimador del parámetro \(\sigma^2\). ¿Cuál es el valor de \(C\)
para que \(T\) sea un estimador insesgado?

\hypertarget{ejercicio-4}{%
\subsection{Ejercicio 4}\label{ejercicio-4}}

Una variable aleatoria \(X\) sigue la distribución de Rayleigh con
parámetro \(\theta >0\) si es una variable aleatoria con valores \(x>0\)
y función de densidad: \[
f(x)=\frac{x}{\theta} e^{-\frac{x^2}{2\theta}}.
\] Hallar el estimador máximo verosímil del parámetro \(\theta\).

\hypertarget{soluciones}{%
\section{Soluciones}\label{soluciones}}

\hypertarget{soluciuxf3n-ejercicio-1}{%
\subsection{Solución ejercicio 1}\label{soluciuxf3n-ejercicio-1}}

Recordemos que \(E(X_i)=\mu\), y que
\(\sigma^2=E(X_i^2)-\left(E(X_i)\right)^2=E(X_i^2)-\mu^2\) luego
\(E(X_i^2)=\sigma^2+\mu^2\) para \(i=1,2,3\). Además al ser
independientes\\
\(E(X_1\cdot X_{2})=E(X_1)\cdot E(X_2)\mu\cdot\mu=\mu^2\) y lo mismo
para \(E(X_3\cdot X_{4})=E(X_5\cdot X_{6})=\mu^2.\)

Ahora

\[
\begin{array}{rl}
E\left(\left(X_{1}-X_{2}\right)^2\right) & =
E\left(X_{1}^2-2\cdot X_1\cdot X_{2}+X_{2}^2\right)\\
& = E\left(X_{1}^2\right)-2\cdot E(X_1\cdot X_{2})+E\left(X_{2}^2\right)\\
& = (\sigma^2+\mu^2)-2\cdot (\mu\cdot\mu)+ (\sigma^2+\mu^2)\\
& =2\cdot\sigma^2. 
\end{array}
\]

Utilizando adecuadamente los cálculos anteriores \[
E\left(C\cdot\bigl({(X_1 -X_2)}^2 +{(X_3 -X_4)}^2 + {(X_5 -X_6)}^2\bigr)\right)=C\cdot 3\cdot 2\cdot \sigma^2.
\]

Luego el valor de \(C\) para que el estimador sea insesgado es la
solución de la ecuación

\[C\cdot 3\cdot 2\cdot \sigma^2=\sigma^2\]

Así que el valor buscado es \(C=\frac16.\)

\hypertarget{soluciuxf3n-ejercicio-2}{%
\subsection{Solución ejercicio 2}\label{soluciuxf3n-ejercicio-2}}

Supongamos que \(\Theta_1\) y \(\Theta_2\) son estimadores sin sesgo de
un parámetro desconocido \(\theta\), con varianzas conocidas
\(\sigma_1^2\) y \(\sigma_2^2\), respectivamente. Demostrar que
\(\Theta =(1-a)\cdot\Theta_1 +a\cdot \Theta_2\) también es insesgado
para cualquier valor de \(a\).

Tenemos que \(E(\Theta_1)=E(\Theta_2)=\theta\) y que
\(Var(\Theta_1)=\sigma_1\) y \(Var(\Theta_2)=\sigma_2\).

Nos piden que demostremos que para cualquier \(a\not=0\)

\[
E(\Theta) =E\left((1-a)\cdot\Theta_1 +a\cdot \Theta_2\right)=\theta.
\]

efectivamente

\(E(\Theta)=(1-a)\cdot E(\Theta_1) +a\cdot E(\Theta_2)=(1-a)\cdot \theta +a\cdot \theta=\theta.\)

\hypertarget{soluciuxf3n-ejercicio-3}{%
\subsection{Solución ejercicio 3}\label{soluciuxf3n-ejercicio-3}}

Como \(X_1,\ldots,X_{2n}\) una muestra aleatoria simple de una variable
aleatoria \(N(\mu,\sigma)\). Sabemos que \(\sum_{i=1}^{2\cdot n} X_i\)
sigue una ley
\(N\left(2\cdot n\cdot\mu, \sqrt{2\cdot n \cdot\sigma^2}\right)\) luego
\(E\left(\sum_{i=1}^{2\cdot n} X_i\right)=2\cdot n\cdot \mu\),
\(Var(\sum_{i=1}^{2\cdot n} X_i)=2\cdot n\cdot \sigma^2\) y
\(E\left(\left(\sum_{i=1}^{2\cdot n}X_i\right)^2\right)=2\cdot n\cdot \sigma^2+2\cdot n\cdot\mu^2.\)
Ademas como \(X_{2\cdot i}\) y \(X_{2\cdot i-1}\) son independientes
\(E(X_{2\cdot i}\cdot X_{2\cdot i-1})=E(X_{2\cdot i})\cdot E(X_{2\cdot i-1})=\mu\cdot \mu=\mu^2.\)

Entonces

\[
\begin{array}{rl}
E(T)&=E\left(C\cdot\left(\left(\sum_{i=1}^{2n} X_i\right)^2- 4 n\sum_{i=1}^{n}
X_{2i} X_{2i-1}\right)\right)\\
& =C\cdot \left(2\cdot n\cdot\sigma^2+(2\cdot n\cdot\mu)^2-4\cdot n\cdot n\cdot \mu^2\right)\\
& = C\cdot2\cdot n\cdot\sigma^2
\end{array}
\]

Por lo que el valor de \(C\) pedido es la solución de la ecuación \[
C\cdot 2\cdot n\cdot\sigma^2=\sigma^2\]

despejando \(C\) obtenemos que el valor buscado es
\(C=\frac{1}{2\cdot n}.\)

\hypertarget{soluciuxf3n-ejercicio-4}{%
\subsection{Solución ejercicio 4}\label{soluciuxf3n-ejercicio-4}}

Sea \(x_1,x_2,\ldots,x_n\) la realización de una más de una variable
aleatoria \(X\) con distribución Rayleigh de parámetro \(\theta >0\). Su
función de verosimilitud es

\[ 
\begin{array}{rl}
L(\theta|x_1,,x_2,\ldots,x_n)&=   f_X(x_1;\theta)\cdot f_X(x_1;\theta)\cdot \ldots \cdot f_x(x_n;\theta)\\
& =\frac{x_1}{\theta}\mathrm{e}^{-\frac{x_1^2}{2\theta}}\cdot\frac{x_2}{\theta} \mathrm{e}^{-\frac{x_2^2}{2\theta}}\cdot\ldots\cdot \frac{x_n}{\theta} \mathrm{e}^{-\frac{x_n^2}{2\theta}}\\
& = \frac{\prod_{i=1}^n x_i}{\theta^n} \cdot \mathrm{e}^{-\frac{ \sum_{i=1}^n x_i^2}{2\theta}}.
\end{array}
\] Buscamos el \(\theta\) que maximiza \(L(\theta|x_1,,x_2,\ldots,x_n)\)
que es el mismo que maximiza su logaritmo, es decir

\[
\ln(L(\theta|x_1,,x_2,\ldots,x_n))=\ln\left(\frac{\prod_{i=1}^n x_i}{\theta^n} \cdot \mathrm{e}^{-\frac{\sum_{i=1}^n x_i^2}{2\theta}}\right)=
\ln\left(\prod_{i=1}^n x_i\right)-\ln(\theta^n)-\frac{ \sum_{i=1}^n x_i^2}{2\theta}.
\]

Ahora derivamos la expresión respecto de \(\theta\), la igualamos a cero
y despejamos \(\theta\)

\[
\ln(L(\theta|x_1,,x_2,\ldots,x_n))'=\left(
\ln\left(\prod_{i=1}^n x_i\right)-\ln(\theta^n)-\frac{ \sum_{i=1}^n x_i^2}{2\theta}\right)'= -\frac{n}{\theta}+\frac{ \sum_{i=1}^n x_i^2}{2\theta^2}=0.
\]

de donde multiplicando los dos términos de la última igualdad por
\(\theta^2.\)

\(-n\cdot \theta+\frac{ \sum_{i=1}^n x_i^2}{2}=0\)

resolviendo la ecuación obtenemos que el estimador máximo verosímil de
\(\theta\) es \[\hat{\theta}= \frac{ \sum_{i=1}^n x_i^2}{2\cdot n}.\]

\end{document}
