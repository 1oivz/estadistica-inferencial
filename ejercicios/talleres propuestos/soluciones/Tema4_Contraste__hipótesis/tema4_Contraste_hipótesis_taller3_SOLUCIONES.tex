% Options for packages loaded elsewhere
\PassOptionsToPackage{unicode}{hyperref}
\PassOptionsToPackage{hyphens}{url}
\PassOptionsToPackage{dvipsnames,svgnames*,x11names*}{xcolor}
%
\documentclass[
]{article}
\usepackage{lmodern}
\usepackage{amssymb,amsmath}
\usepackage{ifxetex,ifluatex}
\ifnum 0\ifxetex 1\fi\ifluatex 1\fi=0 % if pdftex
  \usepackage[T1]{fontenc}
  \usepackage[utf8]{inputenc}
  \usepackage{textcomp} % provide euro and other symbols
\else % if luatex or xetex
  \usepackage{unicode-math}
  \defaultfontfeatures{Scale=MatchLowercase}
  \defaultfontfeatures[\rmfamily]{Ligatures=TeX,Scale=1}
\fi
% Use upquote if available, for straight quotes in verbatim environments
\IfFileExists{upquote.sty}{\usepackage{upquote}}{}
\IfFileExists{microtype.sty}{% use microtype if available
  \usepackage[]{microtype}
  \UseMicrotypeSet[protrusion]{basicmath} % disable protrusion for tt fonts
}{}
\makeatletter
\@ifundefined{KOMAClassName}{% if non-KOMA class
  \IfFileExists{parskip.sty}{%
    \usepackage{parskip}
  }{% else
    \setlength{\parindent}{0pt}
    \setlength{\parskip}{6pt plus 2pt minus 1pt}}
}{% if KOMA class
  \KOMAoptions{parskip=half}}
\makeatother
\usepackage{xcolor}
\IfFileExists{xurl.sty}{\usepackage{xurl}}{} % add URL line breaks if available
\IfFileExists{bookmark.sty}{\usepackage{bookmark}}{\usepackage{hyperref}}
\hypersetup{
  pdftitle={Ejercicios Tema 4 - Contraste hipótesis. Taller 3},
  pdfauthor={Ricardo Alberich, Juan Gabriel Gomila y Arnau Mir},
  colorlinks=true,
  linkcolor=red,
  filecolor=Maroon,
  citecolor=blue,
  urlcolor=blue,
  pdfcreator={LaTeX via pandoc}}
\urlstyle{same} % disable monospaced font for URLs
\usepackage[margin=1in]{geometry}
\usepackage{color}
\usepackage{fancyvrb}
\newcommand{\VerbBar}{|}
\newcommand{\VERB}{\Verb[commandchars=\\\{\}]}
\DefineVerbatimEnvironment{Highlighting}{Verbatim}{commandchars=\\\{\}}
% Add ',fontsize=\small' for more characters per line
\usepackage{framed}
\definecolor{shadecolor}{RGB}{248,248,248}
\newenvironment{Shaded}{\begin{snugshade}}{\end{snugshade}}
\newcommand{\AlertTok}[1]{\textcolor[rgb]{0.94,0.16,0.16}{#1}}
\newcommand{\AnnotationTok}[1]{\textcolor[rgb]{0.56,0.35,0.01}{\textbf{\textit{#1}}}}
\newcommand{\AttributeTok}[1]{\textcolor[rgb]{0.77,0.63,0.00}{#1}}
\newcommand{\BaseNTok}[1]{\textcolor[rgb]{0.00,0.00,0.81}{#1}}
\newcommand{\BuiltInTok}[1]{#1}
\newcommand{\CharTok}[1]{\textcolor[rgb]{0.31,0.60,0.02}{#1}}
\newcommand{\CommentTok}[1]{\textcolor[rgb]{0.56,0.35,0.01}{\textit{#1}}}
\newcommand{\CommentVarTok}[1]{\textcolor[rgb]{0.56,0.35,0.01}{\textbf{\textit{#1}}}}
\newcommand{\ConstantTok}[1]{\textcolor[rgb]{0.00,0.00,0.00}{#1}}
\newcommand{\ControlFlowTok}[1]{\textcolor[rgb]{0.13,0.29,0.53}{\textbf{#1}}}
\newcommand{\DataTypeTok}[1]{\textcolor[rgb]{0.13,0.29,0.53}{#1}}
\newcommand{\DecValTok}[1]{\textcolor[rgb]{0.00,0.00,0.81}{#1}}
\newcommand{\DocumentationTok}[1]{\textcolor[rgb]{0.56,0.35,0.01}{\textbf{\textit{#1}}}}
\newcommand{\ErrorTok}[1]{\textcolor[rgb]{0.64,0.00,0.00}{\textbf{#1}}}
\newcommand{\ExtensionTok}[1]{#1}
\newcommand{\FloatTok}[1]{\textcolor[rgb]{0.00,0.00,0.81}{#1}}
\newcommand{\FunctionTok}[1]{\textcolor[rgb]{0.00,0.00,0.00}{#1}}
\newcommand{\ImportTok}[1]{#1}
\newcommand{\InformationTok}[1]{\textcolor[rgb]{0.56,0.35,0.01}{\textbf{\textit{#1}}}}
\newcommand{\KeywordTok}[1]{\textcolor[rgb]{0.13,0.29,0.53}{\textbf{#1}}}
\newcommand{\NormalTok}[1]{#1}
\newcommand{\OperatorTok}[1]{\textcolor[rgb]{0.81,0.36,0.00}{\textbf{#1}}}
\newcommand{\OtherTok}[1]{\textcolor[rgb]{0.56,0.35,0.01}{#1}}
\newcommand{\PreprocessorTok}[1]{\textcolor[rgb]{0.56,0.35,0.01}{\textit{#1}}}
\newcommand{\RegionMarkerTok}[1]{#1}
\newcommand{\SpecialCharTok}[1]{\textcolor[rgb]{0.00,0.00,0.00}{#1}}
\newcommand{\SpecialStringTok}[1]{\textcolor[rgb]{0.31,0.60,0.02}{#1}}
\newcommand{\StringTok}[1]{\textcolor[rgb]{0.31,0.60,0.02}{#1}}
\newcommand{\VariableTok}[1]{\textcolor[rgb]{0.00,0.00,0.00}{#1}}
\newcommand{\VerbatimStringTok}[1]{\textcolor[rgb]{0.31,0.60,0.02}{#1}}
\newcommand{\WarningTok}[1]{\textcolor[rgb]{0.56,0.35,0.01}{\textbf{\textit{#1}}}}
\usepackage{graphicx}
\makeatletter
\def\maxwidth{\ifdim\Gin@nat@width>\linewidth\linewidth\else\Gin@nat@width\fi}
\def\maxheight{\ifdim\Gin@nat@height>\textheight\textheight\else\Gin@nat@height\fi}
\makeatother
% Scale images if necessary, so that they will not overflow the page
% margins by default, and it is still possible to overwrite the defaults
% using explicit options in \includegraphics[width, height, ...]{}
\setkeys{Gin}{width=\maxwidth,height=\maxheight,keepaspectratio}
% Set default figure placement to htbp
\makeatletter
\def\fps@figure{htbp}
\makeatother
\setlength{\emergencystretch}{3em} % prevent overfull lines
\providecommand{\tightlist}{%
  \setlength{\itemsep}{0pt}\setlength{\parskip}{0pt}}
\setcounter{secnumdepth}{5}
\renewcommand{\contentsname}{Contenidos}

\title{Ejercicios Tema 4 - Contraste hipótesis. Taller 3}
\author{Ricardo Alberich, Juan Gabriel Gomila y Arnau Mir}
\date{Curso completo de estadística inferencial con R y Python}

\begin{document}
\maketitle

{
\hypersetup{linkcolor=blue}
\setcounter{tocdepth}{4}
\tableofcontents
}
\hypertarget{contraste-hipuxf3tesis-taller-3-contrastes-de-dos-paruxe1metros.}{%
\section{Contraste hipótesis taller 3: Contrastes de dos
parámetros.}\label{contraste-hipuxf3tesis-taller-3-contrastes-de-dos-paruxe1metros.}}

\hypertarget{ejercicio-1}{%
\subsection{Ejercicio 1}\label{ejercicio-1}}

Para comparar la producción media de dos procedimientos de fabricación
de cierto producto se toman dos muestras, una con la cantidad producida
durante 25 días con el primer método y otra con la cantidad producida
durante 16 días con el segundo método. Por experiencia se sabe que la
varianza del primer procedimiento es \(\sigma_{1}^2=12\) y al del
segundo \(\sigma_{2}^2=10\). De las muestras obtenemos que
\(\overline{X}_{1}=136\) para el primer procedimiento y
\(\overline{X}_{2}=128\) para el segundo. Si \(\mu_{1}\) y \(\mu_{2}\)
son los valores esperados para cada uno de los procedimientos, calcular
un intervalo de confianza para \(\mu_{1}-\mu_{2}\) al nivel 99\%.

\hypertarget{soluciuxf3n}{%
\subsubsection{Solución}\label{soluciuxf3n}}

Estamos en el caso de un contraste de comparación de medias de muestras
independientes y en el teórico caso de que las varianzas son conocidas.

Contrastaremos:

\[
\left\{
\begin{array}{ll}
H_{0}:\mu_1=\mu_2\\
H_{1}:\mu_1\not= \mu_2 
\end{array}
\right.
\]

El estadístico de contraste es

El \textbf{estadístico de contraste} toma el valor para
\(\alpha=1-0.99=0.01\):

\[z_0=\dfrac{\overline{X}_1-\overline{X}_2}{\sqrt{\frac{\sigma_1^2}{n_1}+\frac{\sigma_2^2}{n_2}}}=\frac{136-128}{\sqrt{\frac{3.4641^2}{25}+\frac{3.1623^2}{16}}}=7.61.\]

El estadístico sigue, aproximadamente, una distribución normal. La
región crítica del contraste al nivel se significación \(\alpha=0.01\)
es rechazar \(H_0\) si \(z_0<z_{\alpha/2}\) o \(z_0>z_{1-\alpha/2}\).
Con nuestros datos

\[z_0=7.61\not< z_{\alpha/2}=z_{0.005}=-2.5758 \mbox{ o } z_0=7.61  > z_{1-\alpha/2}=
z_{0.995}=2.5758\] o lo que es lo mismo rechazamos \(H_0\) si
\(|z_0|=7.61> z_{1-\alpha/2}=2.5758\) lo que en este caso es cierto.

Los cuantiles los hemos calculado con

\begin{Shaded}
\begin{Highlighting}[]
\NormalTok{alpha=}\DecValTok{1}\FloatTok{{-}0.99}
\KeywordTok{qnorm}\NormalTok{(alpha}\OperatorTok{/}\DecValTok{2}\NormalTok{)}
\end{Highlighting}
\end{Shaded}

\begin{verbatim}
## [1] -2.576
\end{verbatim}

\begin{Shaded}
\begin{Highlighting}[]
\OperatorTok{{-}}\KeywordTok{qnorm}\NormalTok{(}\DecValTok{1}\OperatorTok{{-}}\NormalTok{alpha}\OperatorTok{/}\DecValTok{2}\NormalTok{)}
\end{Highlighting}
\end{Shaded}

\begin{verbatim}
## [1] -2.576
\end{verbatim}

\begin{Shaded}
\begin{Highlighting}[]
\KeywordTok{qnorm}\NormalTok{(}\DecValTok{1}\OperatorTok{{-}}\NormalTok{alpha}\OperatorTok{/}\DecValTok{2}\NormalTok{)}
\end{Highlighting}
\end{Shaded}

\begin{verbatim}
## [1] 2.576
\end{verbatim}

El \(p\)-valor de este contraste para la alternativa bilateral es
\(p\)-valor=\(2\cdot P(Z>|z_0|)\) donde \(Z\) es una normal estándar
\(N(\mu=0,\sigma=1)\). Podemos calcularlo con el código

En nuestro caso

\[2\cdot P(Z>|z_0|)= 2\cdot P(Z>|7.61|)=P(Z>7.61)=2\cdot(1-P(Z\leq 7.61 )),\]

lo podemos calcular con R

\begin{Shaded}
\begin{Highlighting}[]
\NormalTok{z0}
\end{Highlighting}
\end{Shaded}

\begin{verbatim}
## [1] 7.61
\end{verbatim}

\begin{Shaded}
\begin{Highlighting}[]
\DecValTok{2}\OperatorTok{*}\NormalTok{(}\DecValTok{1}\OperatorTok{{-}}\KeywordTok{pnorm}\NormalTok{(}\KeywordTok{abs}\NormalTok{(z0)))}
\end{Highlighting}
\end{Shaded}

\begin{verbatim}
## [1] 0.00000000000002731
\end{verbatim}

Es un \(p\)-valor muy pequeño lo que confirma que hay evidencias para
rechazar las hipótesis nula: el rendimiento de los dos métodos de
fabricación no tiene la misma media.

\hypertarget{ejercicio-2}{%
\subsection{Ejercicio 2}\label{ejercicio-2}}

Estamos interesados en comparar la vida media, expresada en horas de dos
tipos de componentes electrónicos. Para ello se toma una muestra de cada
tipo y se obtiene:

\begin{center}
\begin{tabular}{|c|c|c|c|}
\hline Tipo & tamaño & $\overline{x}$ & $\tilde{s}$\\ \hline \hline 1 & 50 & 1260 & 20\\ \hline 2 &
100 & 1240 & 18\\ \hline
\end{tabular}
\end{center}

Calcular un intervalo de confianza para \(\mu_{1}-\mu_{2}\) (\(\mu_{1}\)
esperanza del primer grupo y \(\mu_{2}\) esperanza del segundo grupo) al
nivel 98\% Suponer si es necesario las poblaciones aproximadamente
normales.

\hypertarget{soluciuxf3n-1}{%
\subsubsection{Solución}\label{soluciuxf3n-1}}

En este caso tenemos dos muestras independientes de tamaños \(n1=50\),
\(n2=100\) y estadísticos \(\overline{x}_1=1260\),
\(\overline{x}_2=1240\), \(\tilde{s}_1=20\) y \(\tilde{s}_2=18\).

Cargamos los datos en R

\begin{Shaded}
\begin{Highlighting}[]
\NormalTok{n1=}\DecValTok{50}
\NormalTok{n2=}\DecValTok{120}
\NormalTok{media1=}\DecValTok{1260}
\NormalTok{media2=}\DecValTok{1240}
\NormalTok{desv\_tipica1=}\DecValTok{20}
\NormalTok{desv\_tipica2=}\DecValTok{18}
\NormalTok{f0=desv\_tipica1}\OperatorTok{\^{}}\DecValTok{2}\OperatorTok{/}\NormalTok{desv\_tipica2}\OperatorTok{\^{}}\DecValTok{2} \CommentTok{\# estadístico de contraste}
\NormalTok{f0}
\end{Highlighting}
\end{Shaded}

\begin{verbatim}
## [1] 1.235
\end{verbatim}

Tenemos pues, dos muestras independientes de tamaños muestrales y las
varianzas desconocidas. Haremos un \(t\)-test pero tenemos dos casos:
varianzas desconocidas iguales y varianzas desconocidas distintas.
Primero haremos un test para saber si las varianzas son iguales o
distintas.

El contraste es

\[
\left\{
\begin{array}{ll}
H_{0}:\sigma_1=\sigma_2\\
H_{1}:\sigma_1\not= \sigma_2 
\end{array}
\right.
\]

Se emplea el siguiente \textbf{estadístico de contraste}:

\[
F=\frac{\widetilde{S}_1^2}{\widetilde{S}_2^2}
\] que, si las dos poblaciones son normales y la hipótesis nula
\(H_0:\sigma_1=\sigma_2\) es cierta, tiene distribución \(F\) de Fisher
con grados de libertad \(n_1-1\) y \(n_2-1\).

En nuestro caso
\(f_0=\frac{\tilde{s}_1^2}{\tilde{s}_1^2}= \frac{20^2}{18^2}=1\)

Resolveremos calculando el \(p\)-valor del contraste que este caso es

\[
\begin{array}{l}
\min\{2\cdot P(F_{n_1-1,n_2-1}\leq f_0),2\cdot P(F_{n_1-1,n_2-1}\geq f_0)\}\\=\min\{2\cdot P(F_{50-1,100-1}\leq 1.2346),2\cdot P(F_{50-1,100-1}\geq 1.2346)\}
\end{array}
\]

calcularemos el \(p\)-valor con R

\begin{Shaded}
\begin{Highlighting}[]
\NormalTok{n1=}\DecValTok{50}
\NormalTok{n2=}\DecValTok{100}
\NormalTok{desv\_tipica1=}\DecValTok{20}
\NormalTok{desv\_tipica2=}\DecValTok{18}
\NormalTok{f0=desv\_tipica1}\OperatorTok{\^{}}\DecValTok{2}\OperatorTok{/}\NormalTok{desv\_tipica2}\OperatorTok{\^{}}\DecValTok{2} \CommentTok{\# estadístico de contraste}
\NormalTok{f0}
\end{Highlighting}
\end{Shaded}

\begin{verbatim}
## [1] 1.235
\end{verbatim}

\begin{Shaded}
\begin{Highlighting}[]
\DecValTok{2}\OperatorTok{*}\KeywordTok{pf}\NormalTok{(f0,n1}\DecValTok{{-}1}\NormalTok{,n2}\DecValTok{{-}1}\NormalTok{,}\DataTypeTok{lower.tail =} \OtherTok{TRUE}\NormalTok{)}\CommentTok{\# lower.tail = TRUE es el valor por defecto }
\end{Highlighting}
\end{Shaded}

\begin{verbatim}
## [1] 1.625
\end{verbatim}

\begin{Shaded}
\begin{Highlighting}[]
\DecValTok{2}\OperatorTok{*}\KeywordTok{pf}\NormalTok{(f0,n1}\DecValTok{{-}1}\NormalTok{,n2}\DecValTok{{-}1}\NormalTok{,}\DataTypeTok{lower.tail =} \OtherTok{FALSE}\NormalTok{)}\CommentTok{\# o también 2*(1{-}pf(f0,n1{-}1,n2{-}1)  }
\end{Highlighting}
\end{Shaded}

\begin{verbatim}
## [1] 0.3749
\end{verbatim}

\begin{Shaded}
\begin{Highlighting}[]
\NormalTok{pvalor=}\KeywordTok{min}\NormalTok{(}\DecValTok{2}\OperatorTok{*}\KeywordTok{pf}\NormalTok{(f0,n1}\DecValTok{{-}1}\NormalTok{,n2}\DecValTok{{-}1}\NormalTok{,}\DataTypeTok{lower.tail =} \OtherTok{TRUE}\NormalTok{),}\DecValTok{2}\OperatorTok{*}\KeywordTok{pf}\NormalTok{(f0,n1}\DecValTok{{-}1}\NormalTok{,n2}\DecValTok{{-}1}\NormalTok{,}\DataTypeTok{lower.tail =} \OtherTok{FALSE}\NormalTok{))}
\NormalTok{pvalor}
\end{Highlighting}
\end{Shaded}

\begin{verbatim}
## [1] 0.3749
\end{verbatim}

El \(p\)-valor es alto así que no podemos rechazar la hipótesis nula;
consideraremos las varianzas iguales.

Así pues vamos a contrastar la igualdad de medias contra que son
distintas:

\[
\left\{
\begin{array}{ll}
H_{0}:\mu_1=\mu_2\\
H_{1}:\mu_2\not= \mu_2 
\end{array}
\right..
\]

El estadístico de contraste sigue una ley \(t\) de Student con
\(n_1+n_2-2\) grados de libertad su fórmula es

\[
T=\frac{\overline{X}_1-\overline{X}_2}
{\sqrt{(\frac1{n_1}+\frac1{n_2})\cdot 
\frac{(n_1-1)\widetilde{S}_1^2+(n_2-1)\widetilde{S}_2^2}
{(n_1+n_2-2)}}},
\]

en nuestro caso vale

\[
t_0=\frac{1260-1240}
{\sqrt{(\frac1{50}+\frac1{100})\cdot 
\frac{((50-1)\cdot 20^2+(100-1)\cdot  18^2)}
{(n_1+n_2-2)}}}=7.7264 
\]

con R es

\begin{Shaded}
\begin{Highlighting}[]
\NormalTok{n1 =}\StringTok{ }\DecValTok{50}
\NormalTok{n2 =}\StringTok{ }\DecValTok{120}
\NormalTok{media1 =}\StringTok{ }\DecValTok{1260}
\NormalTok{media2 =}\StringTok{ }\DecValTok{1240}
\NormalTok{desv\_tipica1 =}\StringTok{ }\DecValTok{20}
\NormalTok{desv\_tipica2 =}\StringTok{ }\DecValTok{18}
\NormalTok{t0=(media1}\OperatorTok{{-}}\NormalTok{media2)}\OperatorTok{/}\NormalTok{(}\KeywordTok{sqrt}\NormalTok{((}\DecValTok{1}\OperatorTok{/}\NormalTok{n1}\OperatorTok{+}\DecValTok{1}\OperatorTok{/}\NormalTok{n2)}\OperatorTok{*}\NormalTok{((n1}\DecValTok{{-}1}\NormalTok{)}\OperatorTok{*}\NormalTok{desv\_tipica1}\OperatorTok{+}
\StringTok{                        }\NormalTok{(n2}\DecValTok{{-}1}\NormalTok{)}\OperatorTok{*}\NormalTok{desv\_tipica2}\OperatorTok{\^{}}\DecValTok{2}\NormalTok{)}\OperatorTok{/}\NormalTok{(n1}\OperatorTok{+}\NormalTok{n2}\DecValTok{{-}2}\NormalTok{)))}
\NormalTok{t0}
\end{Highlighting}
\end{Shaded}

\begin{verbatim}
## [1] 7.745
\end{verbatim}

el \(p\)-valor para la alternativa bilateral es
\(2\cdot P(t_{N1+n_2-2}>|t_0|)=2\cdot (1-P(t_{N1+n_2-2}>|t_0|)),\) con R
es

\begin{Shaded}
\begin{Highlighting}[]
\NormalTok{t0}
\end{Highlighting}
\end{Shaded}

\begin{verbatim}
## [1] 7.745
\end{verbatim}

\begin{Shaded}
\begin{Highlighting}[]
\KeywordTok{abs}\NormalTok{(t0)}
\end{Highlighting}
\end{Shaded}

\begin{verbatim}
## [1] 7.745
\end{verbatim}

\begin{Shaded}
\begin{Highlighting}[]
\NormalTok{n1}
\end{Highlighting}
\end{Shaded}

\begin{verbatim}
## [1] 50
\end{verbatim}

\begin{Shaded}
\begin{Highlighting}[]
\NormalTok{n2}
\end{Highlighting}
\end{Shaded}

\begin{verbatim}
## [1] 120
\end{verbatim}

\begin{Shaded}
\begin{Highlighting}[]
\NormalTok{pvalor=}\StringTok{ }\DecValTok{2}\OperatorTok{*}\NormalTok{(}\DecValTok{1}\OperatorTok{{-}}\KeywordTok{pt}\NormalTok{(}\KeywordTok{abs}\NormalTok{(t0),}\DataTypeTok{df=}\NormalTok{n1}\OperatorTok{+}\NormalTok{n2}\DecValTok{{-}2}\NormalTok{))}
\NormalTok{pvalor}
\end{Highlighting}
\end{Shaded}

\begin{verbatim}
## [1] 0.0000000000008566
\end{verbatim}

EL \(p\)-valor es extremadamente pequeño hay evidencias en contra de la
hipótesis nula de igualdad de medias contra la hipótesis alternativa de
que son distintas.

El intervalo de confianza al nivel \(1-\alpha=0.95\) es

\begin{eqnarray*}
\left(&&
\overline{x}_1-\overline{x}_2- t_{n_1+n_2-2,1-\frac{\alpha}{2}}\cdot  \sqrt{\frac{(n_1-1)\cdot \widetilde{S}_1^2+(n_2-1)\cdot\widetilde{S}_2^2}{(n_1+n_2-2)}}\right. \\
&,&
\left.\overline{x}_1-\overline{x}_2+ t_{n_1+n_2-2,1-\frac{\alpha}{2}}\cdot  \sqrt{\frac{(n_1-1)\cdot \widetilde{S}_1^2+(n_2-1)\cdot\widetilde{S}_2^2}
{(n_1+n_2-2)}}
\right)
\end{eqnarray*}

lo calculamos con R

\begin{Shaded}
\begin{Highlighting}[]
\NormalTok{n1=}\DecValTok{50}
\NormalTok{n2=}\DecValTok{120}
\NormalTok{media1=}\DecValTok{1260}
\NormalTok{media2=}\DecValTok{1240}
\NormalTok{desv\_tipica1=}\DecValTok{20}
\NormalTok{desv\_tipica2=}\DecValTok{18}
\NormalTok{alpha=}\DecValTok{1}\FloatTok{{-}0.98}
\KeywordTok{qt}\NormalTok{(}\DecValTok{1}\OperatorTok{{-}}\NormalTok{alpha}\OperatorTok{/}\DecValTok{2}\NormalTok{,}\DataTypeTok{df=}\NormalTok{n1}\OperatorTok{+}\NormalTok{n2}\DecValTok{{-}2}\NormalTok{)}
\end{Highlighting}
\end{Shaded}

\begin{verbatim}
## [1] 2.349
\end{verbatim}

\begin{Shaded}
\begin{Highlighting}[]
\NormalTok{IC=}\KeywordTok{c}\NormalTok{(media1}\OperatorTok{{-}}\NormalTok{media2}\OperatorTok{{-}}\StringTok{ }\KeywordTok{qt}\NormalTok{(}\DecValTok{1}\OperatorTok{{-}}\NormalTok{alpha}\OperatorTok{/}\DecValTok{2}\NormalTok{,}\DataTypeTok{df=}\NormalTok{n1}\OperatorTok{+}\NormalTok{n2}\DecValTok{{-}2}\NormalTok{)}
     \OperatorTok{*}\KeywordTok{sqrt}\NormalTok{((}\DecValTok{1}\OperatorTok{/}\NormalTok{n1}\OperatorTok{+}\DecValTok{1}\OperatorTok{/}\NormalTok{n2)}\OperatorTok{*}\NormalTok{((n1}\DecValTok{{-}1}\NormalTok{)}\OperatorTok{*}\NormalTok{desv\_tipica1}\OperatorTok{+}\NormalTok{(n2}\DecValTok{{-}1}\NormalTok{)}\OperatorTok{*}\NormalTok{desv\_tipica2}\OperatorTok{\^{}}\DecValTok{2}\NormalTok{)}\OperatorTok{/}\NormalTok{(n1}\OperatorTok{+}\NormalTok{n2}\DecValTok{{-}2}\NormalTok{)),}
\NormalTok{     media1}\OperatorTok{{-}}\NormalTok{media2}\OperatorTok{+}\StringTok{ }\KeywordTok{qt}\NormalTok{(}\DecValTok{1}\OperatorTok{{-}}\NormalTok{alpha}\OperatorTok{/}\DecValTok{2}\NormalTok{,}\DataTypeTok{df=}\NormalTok{n1}\OperatorTok{+}\NormalTok{n2}\DecValTok{{-}2}\NormalTok{)}
     \OperatorTok{*}\KeywordTok{sqrt}\NormalTok{((}\DecValTok{1}\OperatorTok{/}\NormalTok{n1}\OperatorTok{+}\DecValTok{1}\OperatorTok{/}\NormalTok{n2)}\OperatorTok{*}\NormalTok{((n1}\DecValTok{{-}1}\NormalTok{)}\OperatorTok{*}\NormalTok{desv\_tipica1}\OperatorTok{+}\NormalTok{(n2}\DecValTok{{-}1}\NormalTok{)}\OperatorTok{*}\NormalTok{desv\_tipica2}\OperatorTok{\^{}}\DecValTok{2}\NormalTok{)}\OperatorTok{/}\NormalTok{(n1}\OperatorTok{+}\NormalTok{n2}\DecValTok{{-}2}\NormalTok{)))}
\NormalTok{IC}
\end{Highlighting}
\end{Shaded}

\begin{verbatim}
## [1] 13.94 26.06
\end{verbatim}

La diferencia de medias \(\mu_1-\mu_2\) verdadera se encontrará en el
intervalo \((13.9351, 26.0649)\) al nivel de confianza del \(95\%\). La
\(media_1\) es claramente más grande que la dos, en al menos \(14\)

\hypertarget{ejercicio-3}{%
\subsection{Ejercicio 3}\label{ejercicio-3}}

Para reducir la concentración de ácido úrico en la sangre se prueban dos
drogas. La primera se aplica a un grupo de 8 pacientes y la segunda a un
grupo de 10. Las disminuciones observadas en las concentraciones de
ácido úrico de los distintos pacientes expresadas en tantos por cien de
concentración después de aplicado el tratamiento son:

\begin{center}
\begin{tabular}{|c|c|c|c|c|c|c|c|c|c|c|}
droga 1 & 20 & 12 & 16 & 18 & 13 & 22 & 15 & 20\\ \hline droga 2 & 17 & 14 & 12 & 10 & 15 &
13 & 9 & 19 & 20 & 11
\end{tabular}
\end{center}

Suponer que las reducciones de ácido úrico siguen una distribución
normal son independientes

Contrastar la igualdad de medias contra que la droga 1 es mejor (menor
media) que la droga 2. Resolver el test en los dos casos varianzas
iguales y varianzas distintas. Calcular el intervalo de confianza
asociado al contraste

\hypertarget{soluciuxf3n-2}{%
\subsubsection{Solución}\label{soluciuxf3n-2}}

Denotando la droga 1 y 2 con el mismo subíndice el contraste es:

\[
\left\{
\begin{array}{ll}
H_{0}:\mu_1=\mu_2\\
H_{1}:\mu_1 < \mu_2 
\end{array}
\right..
\]

Carguemos los datos y resolvamos con R

\begin{Shaded}
\begin{Highlighting}[]
\NormalTok{droga1=}\KeywordTok{c}\NormalTok{(}\DecValTok{20}\NormalTok{,}\DecValTok{12}\NormalTok{,}\DecValTok{16}\NormalTok{,}\DecValTok{18}\NormalTok{,}\DecValTok{13}\NormalTok{,}\DecValTok{22}\NormalTok{,}\DecValTok{15}\NormalTok{,}\DecValTok{20}\NormalTok{)}
\NormalTok{droga2=}\KeywordTok{c}\NormalTok{(}\DecValTok{17}\NormalTok{,}\DecValTok{14}\NormalTok{,}\DecValTok{12}\NormalTok{,}\DecValTok{10}\NormalTok{,}\DecValTok{15}\NormalTok{,}\DecValTok{13}\NormalTok{,}\DecValTok{9}\NormalTok{,}\DecValTok{19}\NormalTok{,}\DecValTok{20}\NormalTok{,}\DecValTok{11}\NormalTok{)}
\end{Highlighting}
\end{Shaded}

El contraste para el caso de varianzas iguales es

\begin{Shaded}
\begin{Highlighting}[]
\KeywordTok{t.test}\NormalTok{(droga1,droga2,}\DataTypeTok{var.equal =} \OtherTok{TRUE}\NormalTok{,}\DataTypeTok{alternative =} \StringTok{"less"}\NormalTok{,}\DataTypeTok{conf.level=}\FloatTok{0.9}\NormalTok{)}
\end{Highlighting}
\end{Shaded}

\begin{verbatim}
## 
##  Two Sample t-test
## 
## data:  droga1 and droga2
## t = 1.7, df = 16, p-value = 0.9
## alternative hypothesis: true difference in means is less than 0
## 90 percent confidence interval:
##  -Inf 5.33
## sample estimates:
## mean of x mean of y 
##        17        14
\end{verbatim}

\begin{Shaded}
\begin{Highlighting}[]
\NormalTok{p\_valor=}\KeywordTok{t.test}\NormalTok{(droga1,droga2,}\DataTypeTok{var.equal =} \OtherTok{TRUE}\NormalTok{,}\DataTypeTok{alternative =} \StringTok{"less"}\NormalTok{,}\DataTypeTok{conf.level=}\FloatTok{0.9}\NormalTok{)}\OperatorTok{$}\NormalTok{p.value}
\NormalTok{p\_valor}
\end{Highlighting}
\end{Shaded}

\begin{verbatim}
## [1] 0.9478
\end{verbatim}

\begin{Shaded}
\begin{Highlighting}[]
\NormalTok{IC=}\KeywordTok{t.test}\NormalTok{(droga1,droga2,}\DataTypeTok{var.equal =} \OtherTok{TRUE}\NormalTok{,}\DataTypeTok{alternative =} \StringTok{"less"}\NormalTok{,}\DataTypeTok{conf.level=}\FloatTok{0.9}\NormalTok{)}\OperatorTok{$}\NormalTok{conf.int}
\NormalTok{IC}
\end{Highlighting}
\end{Shaded}

\begin{verbatim}
## [1] -Inf 5.33
## attr(,"conf.level")
## [1] 0.9
\end{verbatim}

El \(p\)-valor es 0.9478 muy grande no hay evidencias para aceptar que
la droga1 reduce en media el ácido úrico más que la droga 2.

El intervalo de confianza para la diferencia de medias al nivel de
confianza del \(90\%\) es \((-\infty, 5.3298)\)

Ahora el mismo test pero suponiendo varianzas distintas.

\begin{Shaded}
\begin{Highlighting}[]
\KeywordTok{t.test}\NormalTok{(droga1,droga2,}\DataTypeTok{var.equal =} \OtherTok{FALSE}\NormalTok{,}\DataTypeTok{alternative =} \StringTok{"less"}\NormalTok{,}\DataTypeTok{conf.level=}\FloatTok{0.9}\NormalTok{)}
\end{Highlighting}
\end{Shaded}

\begin{verbatim}
## 
##  Welch Two Sample t-test
## 
## data:  droga1 and droga2
## t = 1.7, df = 15, p-value = 0.9
## alternative hypothesis: true difference in means is less than 0
## 90 percent confidence interval:
##   -Inf 5.322
## sample estimates:
## mean of x mean of y 
##        17        14
\end{verbatim}

\begin{Shaded}
\begin{Highlighting}[]
\NormalTok{p\_valor=}\KeywordTok{t.test}\NormalTok{(droga1,droga2,}\DataTypeTok{var.equal =} \OtherTok{FALSE}\NormalTok{,}\DataTypeTok{alternative =} \StringTok{"less"}\NormalTok{,}\DataTypeTok{conf.level=}\FloatTok{0.9}\NormalTok{)}\OperatorTok{$}\NormalTok{p.value}
\NormalTok{p\_valor}
\end{Highlighting}
\end{Shaded}

\begin{verbatim}
## [1] 0.9482
\end{verbatim}

\begin{Shaded}
\begin{Highlighting}[]
\NormalTok{IC=}\KeywordTok{t.test}\NormalTok{(droga1,droga2,}\DataTypeTok{var.equal =} \OtherTok{FALSE}\NormalTok{,}\DataTypeTok{alternative =} \StringTok{"less"}\NormalTok{,}\DataTypeTok{conf.level=}\FloatTok{0.9}\NormalTok{)}\OperatorTok{$}\NormalTok{conf.int}
\NormalTok{IC}
\end{Highlighting}
\end{Shaded}

\begin{verbatim}
## [1]  -Inf 5.322
## attr(,"conf.level")
## [1] 0.9
\end{verbatim}

El \(p\)-valor es 0.9482 muy grande no hay evidencias para aceptar que
la droga1 reduce de media el ácido úrico más que la droga 2.

El intervalo de confianza para la diferencia de medias al nivel de
confianza del \(90\%\) es \((-\infty, 5.3219)\)

Por último podemos hacer el test de igualdad de varianzas

\begin{Shaded}
\begin{Highlighting}[]
\KeywordTok{var.test}\NormalTok{(droga1,droga2)}
\end{Highlighting}
\end{Shaded}

\begin{verbatim}
## 
##  F test to compare two variances
## 
## data:  droga1 and droga2
## F = 0.92, num df = 7, denom df = 9, p-value = 0.9
## alternative hypothesis: true ratio of variances is not equal to 1
## 95 percent confidence interval:
##  0.2188 4.4295
## sample estimates:
## ratio of variances 
##             0.9184
\end{verbatim}

el \(p\)-valor es alto, no hay evidencias para poder rechazar que las
varianzas sean iguales

\hypertarget{ejercicio-4}{%
\subsection{Ejercicio 4}\label{ejercicio-4}}

Para comparar la dureza media de dos tipos de aleaciones (tipo 1 y tipo
2) se hacen 20 pruebas de dureza con la de tipo 1 y 25 con la de tipo 2.
Obteniéndose los resultados siguientes:

\begin{Shaded}
\begin{Highlighting}[]
\KeywordTok{set.seed}\NormalTok{(}\DecValTok{345}\NormalTok{)}
\NormalTok{aleacion1=}\KeywordTok{round}\NormalTok{(}\FloatTok{0.2}\OperatorTok{*}\NormalTok{(}\KeywordTok{rnorm}\NormalTok{(}\DecValTok{20}\NormalTok{))}\OperatorTok{+}\FloatTok{18.2}\NormalTok{,}\DecValTok{2}\NormalTok{)}
\NormalTok{aleacion2=}\KeywordTok{round}\NormalTok{(}\FloatTok{0.5}\OperatorTok{*}\NormalTok{(}\KeywordTok{rnorm}\NormalTok{(}\DecValTok{25}\NormalTok{))}\OperatorTok{+}\FloatTok{17.8}\NormalTok{,}\DecValTok{2}\NormalTok{)}
\end{Highlighting}
\end{Shaded}

\begin{Shaded}
\begin{Highlighting}[]
\NormalTok{aleacion1=}\KeywordTok{c}\NormalTok{(}\FloatTok{18.04}\NormalTok{,}\FloatTok{18.14}\NormalTok{,}\FloatTok{18.17}\NormalTok{,}\FloatTok{18.14}\NormalTok{,}\FloatTok{18.19}\NormalTok{,}\FloatTok{18.07}\NormalTok{,}\FloatTok{18.01}\NormalTok{,}\FloatTok{18.54}\NormalTok{,}
            \FloatTok{18.53}\NormalTok{,}\FloatTok{18.56}\NormalTok{,}\FloatTok{18.57}\NormalTok{,}\FloatTok{17.92}\NormalTok{,}\FloatTok{18.03}\NormalTok{,}\FloatTok{18.26}\NormalTok{,}\FloatTok{18.38}\NormalTok{,}\FloatTok{17.92}\NormalTok{,}
            \FloatTok{18.31}\NormalTok{,}\FloatTok{18.41}\NormalTok{,}\DecValTok{18}\NormalTok{,}\FloatTok{18.26}\NormalTok{)}
\NormalTok{aleacion2=}\KeywordTok{c}\NormalTok{(}\FloatTok{18.02}\NormalTok{,}\FloatTok{18.21}\NormalTok{,}\FloatTok{16.51}\NormalTok{,}\FloatTok{17.21}\NormalTok{,}\FloatTok{17.85}\NormalTok{,}\FloatTok{18.24}\NormalTok{,}\FloatTok{17.48}\NormalTok{,}\FloatTok{17.28}\NormalTok{,}
            \FloatTok{17.51}\NormalTok{,}\FloatTok{17.51}\NormalTok{,}\FloatTok{17.43}\NormalTok{,}\FloatTok{18.14}\NormalTok{,}\FloatTok{17.32}\NormalTok{,}\FloatTok{17.11}\NormalTok{,}\FloatTok{17.55}\NormalTok{,}\FloatTok{17.49}\NormalTok{,}
            \FloatTok{18.27}\NormalTok{,}\FloatTok{17.92}\NormalTok{,}\FloatTok{18.14}\NormalTok{,}\FloatTok{18.52}\NormalTok{,}\FloatTok{18.12}\NormalTok{,}\FloatTok{18.22}\NormalTok{,}\FloatTok{17.37}\NormalTok{,}\FloatTok{17.91}\NormalTok{,}
            \FloatTok{17.77}\NormalTok{)}
\NormalTok{media1=}\KeywordTok{mean}\NormalTok{(aleacion1)}
\NormalTok{media1}
\end{Highlighting}
\end{Shaded}

\begin{verbatim}
## [1] 18.22
\end{verbatim}

\begin{Shaded}
\begin{Highlighting}[]
\NormalTok{sd1=}\KeywordTok{sd}\NormalTok{(aleacion1)}
\NormalTok{sd1}
\end{Highlighting}
\end{Shaded}

\begin{verbatim}
## [1] 0.2163
\end{verbatim}

\begin{Shaded}
\begin{Highlighting}[]
\NormalTok{media2=}\KeywordTok{mean}\NormalTok{(aleacion2)}
\NormalTok{media2}
\end{Highlighting}
\end{Shaded}

\begin{verbatim}
## [1] 17.72
\end{verbatim}

\begin{Shaded}
\begin{Highlighting}[]
\NormalTok{sd2=}\KeywordTok{sd}\NormalTok{(aleacion2)}
\NormalTok{sd2}
\end{Highlighting}
\end{Shaded}

\begin{verbatim}
## [1] 0.4693
\end{verbatim}

\[\overline{X}_{1}=18.2225,\quad S_{1}=0.2163 \mbox{ y}\]

\[\overline{X}_{1}=17.724,\quad S_{2}=0.4693\]

Suponer que la población de las durezas es normal y que las desviaciones
típicas no son iguales.

Contrastar que las medias de las durezas son iguales contra que son
distintas. Calcular un intervalo de confianza para \(\mu_{1}-\mu_{2}\)
al nivel de significación del \(95\%\).

Haced lo mismo si las varianzas son distintas.

\hypertarget{soluciuxf3n-3}{%
\subsubsection{Solución}\label{soluciuxf3n-3}}

Lo resolvemos solo con R sin más comentarios

Cargamos datos

\begin{Shaded}
\begin{Highlighting}[]
\NormalTok{n1=}\DecValTok{20}
\NormalTok{n2=}\DecValTok{20}
\NormalTok{aleacion1}
\end{Highlighting}
\end{Shaded}

\begin{verbatim}
##  [1] 18.04 18.14 18.17 18.14 18.19 18.07 18.01 18.54 18.53 18.56 18.57 17.92
## [13] 18.03 18.26 18.38 17.92 18.31 18.41 18.00 18.26
\end{verbatim}

\begin{Shaded}
\begin{Highlighting}[]
\NormalTok{aleacion2}
\end{Highlighting}
\end{Shaded}

\begin{verbatim}
##  [1] 18.02 18.21 16.51 17.21 17.85 18.24 17.48 17.28 17.51 17.51 17.43 18.14
## [13] 17.32 17.11 17.55 17.49 18.27 17.92 18.14 18.52 18.12 18.22 17.37 17.91
## [25] 17.77
\end{verbatim}

Contraste

\[
\left\{
\begin{array}{ll}
H_0:&\mu_1=\mu_2\\ 
H_1:&\mu_1\not=\mu_2
\end{array}
\right.
\]

\begin{Shaded}
\begin{Highlighting}[]
\KeywordTok{t.test}\NormalTok{(aleacion1,aleacion2,}\DataTypeTok{alternative =} \StringTok{"two.sided"}\NormalTok{,}\DataTypeTok{var.equal =}\OtherTok{FALSE}\NormalTok{,}\DataTypeTok{conf.level=}\FloatTok{0.95}\NormalTok{)}
\end{Highlighting}
\end{Shaded}

\begin{verbatim}
## 
##  Welch Two Sample t-test
## 
## data:  aleacion1 and aleacion2
## t = 4.7, df = 35, p-value = 0.00004
## alternative hypothesis: true difference in means is not equal to 0
## 95 percent confidence interval:
##  0.2842 0.7128
## sample estimates:
## mean of x mean of y 
##     18.22     17.72
\end{verbatim}

\begin{Shaded}
\begin{Highlighting}[]
\KeywordTok{t.test}\NormalTok{(aleacion1,aleacion2,}\DataTypeTok{alternative =} \StringTok{"two.sided"}\NormalTok{,}\DataTypeTok{var.equal =}\OtherTok{TRUE}\NormalTok{,}\DataTypeTok{conf.level=}\FloatTok{0.95}\NormalTok{)}
\end{Highlighting}
\end{Shaded}

\begin{verbatim}
## 
##  Two Sample t-test
## 
## data:  aleacion1 and aleacion2
## t = 4.4, df = 43, p-value = 0.00007
## alternative hypothesis: true difference in means is not equal to 0
## 95 percent confidence interval:
##  0.2692 0.7278
## sample estimates:
## mean of x mean of y 
##     18.22     17.72
\end{verbatim}

Son las varianzas iguales

\[
\left\{
\begin{array}{ll}
H_0:&\sigma^2_1=\sigma^2_2\\ 
H_1:&\sigma^2_1\not=\sigma^2_2
\end{array}
\right.
\]

\begin{Shaded}
\begin{Highlighting}[]
\KeywordTok{var.test}\NormalTok{(aleacion1,aleacion2,}\DataTypeTok{ratio =} \DecValTok{1}\NormalTok{,}\DataTypeTok{conf.level =} \FloatTok{0.95}\NormalTok{,}\DataTypeTok{alternative =} \StringTok{"two.sided"}\NormalTok{)}
\end{Highlighting}
\end{Shaded}

\begin{verbatim}
## 
##  F test to compare two variances
## 
## data:  aleacion1 and aleacion2
## F = 0.21, num df = 19, denom df = 24, p-value = 0.001
## alternative hypothesis: true ratio of variances is not equal to 1
## 95 percent confidence interval:
##  0.09062 0.52117
## sample estimates:
## ratio of variances 
##             0.2125
\end{verbatim}

\begin{Shaded}
\begin{Highlighting}[]
\KeywordTok{var}\NormalTok{(aleacion1)}\OperatorTok{/}\KeywordTok{var}\NormalTok{(aleacion2)}
\end{Highlighting}
\end{Shaded}

\begin{verbatim}
## [1] 0.2125
\end{verbatim}

\begin{Shaded}
\begin{Highlighting}[]
\KeywordTok{var.test}\NormalTok{(aleacion1,aleacion2,}\DataTypeTok{ratio =} \DecValTok{1}\NormalTok{,}\DataTypeTok{conf.level =} \FloatTok{0.95}\NormalTok{,}\DataTypeTok{alternative =} \StringTok{"two.sided"}\NormalTok{)}\OperatorTok{$}\NormalTok{statistic}
\end{Highlighting}
\end{Shaded}

\begin{verbatim}
##      F 
## 0.2125
\end{verbatim}

\hypertarget{ejercicio-5}{%
\subsection{Ejercicio 5}\label{ejercicio-5}}

Se encuestó a dos muestras independientes de empresas, en las islas de
Ibiza y otra en Mallorca, sobre si utilizaban sistemas de almacenamiento
en la nube. La encuesta de Ibiza tuvo un tamaño \(n_1=500\) y \(200\)
usuarios de la nube, mientras que en Mallorca se encuestaron a
\(n_2=750\) y se obtuvo un resultado de \(210\) usuarios.

se pide:

\begin{enumerate}
\def\labelenumi{\arabic{enumi}.}
\tightlist
\item
  Construir una matriz 2 por 2 que contenga en filas los valores de
  Ibiza y Mallorca y por columnas las respuestas Sí y No
\item
  Con la función \texttt{prop.test} y el ` contrastar si las
  proporciones por islas son iguales o distintas.\\
\item
  Resolver el contraste con el \(p\)-valor y obtener e interpretar un
  los intervalos de confianza del 95\% para la \emph{comparación de las
  proporciones} (!cuidado con el orden¡).
\end{enumerate}

\hypertarget{soluciuxf3n-4}{%
\subsubsection{Solución}\label{soluciuxf3n-4}}

\begin{Shaded}
\begin{Highlighting}[]
\NormalTok{datos=}\KeywordTok{matrix}\NormalTok{(}\KeywordTok{c}\NormalTok{(}\DecValTok{200}\NormalTok{,}\DecValTok{500{-}200}\NormalTok{,}\DecValTok{210}\NormalTok{,}\DecValTok{750{-}210}\NormalTok{),}\DataTypeTok{nrow=}\DecValTok{2}\NormalTok{,}\DataTypeTok{byrow =} \OtherTok{TRUE}\NormalTok{)}
\KeywordTok{dimnames}\NormalTok{(datos)=}\KeywordTok{list}\NormalTok{(}\KeywordTok{c}\NormalTok{(}\StringTok{"Ibiza"}\NormalTok{,}\StringTok{"Mallorca"}\NormalTok{),}\KeywordTok{c}\NormalTok{(}\StringTok{"Sí"}\NormalTok{,}\StringTok{"No"}\NormalTok{))}
\NormalTok{datos}
\end{Highlighting}
\end{Shaded}

\begin{verbatim}
##           Sí  No
## Ibiza    200 300
## Mallorca 210 540
\end{verbatim}

\begin{Shaded}
\begin{Highlighting}[]
\KeywordTok{prop.test}\NormalTok{(datos)}
\end{Highlighting}
\end{Shaded}

\begin{verbatim}
## 
##  2-sample test for equality of proportions with continuity correction
## 
## data:  datos
## X-squared = 19, df = 1, p-value = 0.00001
## alternative hypothesis: two.sided
## 95 percent confidence interval:
##  0.0647 0.1753
## sample estimates:
## prop 1 prop 2 
##   0.40   0.28
\end{verbatim}

El test exacto de odds-ratio se calcula con la función

\begin{Shaded}
\begin{Highlighting}[]
\KeywordTok{fisher.test}\NormalTok{(datos)}
\end{Highlighting}
\end{Shaded}

\begin{verbatim}
## 
##  Fisher's Exact Test for Count Data
## 
## data:  datos
## p-value = 0.00001
## alternative hypothesis: true odds ratio is not equal to 1
## 95 percent confidence interval:
##  1.339 2.194
## sample estimates:
## odds ratio 
##      1.714
\end{verbatim}

En ambos casos se rechaza la hipótesis nula.

El \(p-valor\) calculado de forma directa y como resultado del
\texttt{fisher.test} es aproxcimadamente el mismo(las discrepancias son
pequeñas, posiblemente debidas a redondeos de las funciones\ldots)

\begin{Shaded}
\begin{Highlighting}[]
\KeywordTok{fisher.test}\NormalTok{(datos)}\OperatorTok{$}\NormalTok{p.value}
\end{Highlighting}
\end{Shaded}

\begin{verbatim}
## [1] 0.00001222
\end{verbatim}

\begin{Shaded}
\begin{Highlighting}[]
\DecValTok{2}\OperatorTok{*}\KeywordTok{min}\NormalTok{(}\KeywordTok{phyper}\NormalTok{(}\DecValTok{200}\NormalTok{,}\DecValTok{500}\NormalTok{,}\DecValTok{750}\NormalTok{,}\DecValTok{410}\NormalTok{,}\DataTypeTok{lower.tail =} \OtherTok{TRUE}\NormalTok{),}
            \KeywordTok{phyper}\NormalTok{(}\DecValTok{200}\NormalTok{,}\DecValTok{500}\NormalTok{,}\DecValTok{750}\NormalTok{,}\DecValTok{410}\NormalTok{,}\DataTypeTok{lower.tail =}\OtherTok{FALSE}\NormalTok{))}
\end{Highlighting}
\end{Shaded}

\begin{verbatim}
## [1] 0.000007776
\end{verbatim}

\begin{Shaded}
\begin{Highlighting}[]
\KeywordTok{round}\NormalTok{(}\DecValTok{2}\OperatorTok{*}\KeywordTok{min}\NormalTok{(}\KeywordTok{phyper}\NormalTok{(}\DecValTok{200}\NormalTok{,}\DecValTok{500}\NormalTok{,}\DecValTok{750}\NormalTok{,}\DecValTok{410}\NormalTok{,}\DataTypeTok{lower.tail =} \OtherTok{TRUE}\NormalTok{),}
            \KeywordTok{phyper}\NormalTok{(}\DecValTok{200}\NormalTok{,}\DecValTok{500}\NormalTok{,}\DecValTok{750}\NormalTok{,}\DecValTok{410}\NormalTok{,}\DataTypeTok{lower.tail =}\OtherTok{FALSE}\NormalTok{)),}\DecValTok{5}\NormalTok{)}
\end{Highlighting}
\end{Shaded}

\begin{verbatim}
## [1] 0.00001
\end{verbatim}

El intervalo de confianza es para la odds-ratio
\(\frac{\frac{p1}{1-p1}}{\frac{p2}{1-p2}}\). Así que el intervalo de
confianza debe contener a 1 para que las proporciones sean iguales.
\href{https://joanby.github.io/bookdown-estadistica-inferencial/contrastes-de-hip\%C3\%B3tesis-param\%C3\%A9tricos.html\#contrastes-para-dos-proporciones-p_1-y-p_2}{Consultad
Bookdown Estadística Inferencial capítulo 4 sección 8}.

\hypertarget{ejercicio-6}{%
\subsection{Ejercicio 6}\label{ejercicio-6}}

Se pregunta a un grupo de 100 personas elegido al azar asiste a un
\emph{webinar} sobre tecnología para la banca. Antes de la conferencia
se les pregunta si consideran que Internet es segura para la banca,
después de la conferencia se les vuelve a preguntar cual es su opinión.
Los resultados fueron los siguientes:

\begin{center}
\begin{tabular}{|c|c|cc|}
\cline{3-4}
     \multicolumn{2}{c|}{}& \multicolumn{2}{|c|} {Después}\\\cline{3-4}
   \multicolumn{2}{c|}{} & Sí Segura & No Segura \\\hline
Antes & Sí  Segura &  50 &  20 \\
    & No Segura   &  15 & 15
\\\hline
\end{tabular}
\end{center}

Contrastar, calculando el \(p\)-valor, si ha cambiado (en cualquier
sentido) la proporción de los asistentes que consideran que Internet es
segura para la banca.

\hypertarget{soluciuxf3n-5}{%
\subsubsection{Solución}\label{soluciuxf3n-5}}

Es un contraste de comparación de proporciones emparejadas con R se
puede resolver, entre otras funciones, con las dos funciones siguientes

\begin{Shaded}
\begin{Highlighting}[]
\NormalTok{datos=}\KeywordTok{matrix}\NormalTok{(}\KeywordTok{c}\NormalTok{(}\DecValTok{50}\NormalTok{,}\DecValTok{20}\NormalTok{, }\DecValTok{15}\NormalTok{,}\DecValTok{15}\NormalTok{),}\DataTypeTok{nrow=}\DecValTok{2}\NormalTok{,}\DataTypeTok{byrow =} \OtherTok{TRUE}\NormalTok{)}
\NormalTok{datos}
\end{Highlighting}
\end{Shaded}

\begin{verbatim}
##      [,1] [,2]
## [1,]   50   20
## [2,]   15   15
\end{verbatim}

\begin{Shaded}
\begin{Highlighting}[]
\KeywordTok{dimnames}\NormalTok{(datos)=}\KeywordTok{list}\NormalTok{(}\KeywordTok{c}\NormalTok{(}\StringTok{"Antes\_Si"}\NormalTok{,}\StringTok{"Antes\_No\_Segura"}\NormalTok{),}\KeywordTok{c}\NormalTok{(}\StringTok{"Despues\_Si"}\NormalTok{,}\StringTok{"Despues\_No"}\NormalTok{))}
\NormalTok{datos}
\end{Highlighting}
\end{Shaded}

\begin{verbatim}
##                 Despues_Si Despues_No
## Antes_Si                50         20
## Antes_No_Segura         15         15
\end{verbatim}

\begin{Shaded}
\begin{Highlighting}[]
\KeywordTok{mcnemar.test}\NormalTok{(datos)}
\end{Highlighting}
\end{Shaded}

\begin{verbatim}
## 
##  McNemar's Chi-squared test with continuity correction
## 
## data:  datos
## McNemar's chi-squared = 0.46, df = 1, p-value = 0.5
\end{verbatim}

El \(p\)-valor es alto, no hay evidencias contra la igualdad de las
proporciones entre antes y después del seminario. Por lo tanto los
asistentes (en proporciones) no han cambiado de opinión.

\end{document}
